\subsection{Implementation Options/Designs: }
\vspace {5 mm}
\textbf{Summary of Project Description and Specification}
\newline
\newline
The description of the project to be completed is rather vague, however, it does state that the format of the solution is to be a database which can either be implemented online (i.e. on a website) or offline (i.e. run on a local machine as an executable program). There are no specific details as to which language, layout or structure etc, are to be used when creating the solution. There are also no details as to which Operating System the solution should be created for and whether additional software/hardware is allowed. Considering this we have ‘taken it open ourselves’ to discuss and choose what we thought was a suitable target platform and have also discussed availability of software/hardware needed to create the solution.
During an initial meeting with our client concerning the problem specification/requirements it has become clear that the preferred solution is to create a website with a database backbone (which can also be seen in the Problem Specification). This can be implemented in many different styles/languages which we have also discussed extensively. 
We have created the Problem Specification with the intent that at each stage, the structure and format of the solution allows successive stages to be completed without changing the entire structure too much. This will make the solution easily upgradable for the members of our group or possibly external groups/people in the future. This therefore means that to keep to this method of creating the solution we need to find a language that has all the functionality to complete all of the requirements and also work easily with a database and with some sort of web programming language such as HTML .
\newline
\newline
\textbf{Language Choices}
\newline
\newline
Taking into account the requirements and also our experience as a group regarding programming languages, our Technical Officer suggested that we use the language Django as the backbone of the solution with a HTML webpage implementation for the GUI. We came to this outcome as the Django language is very compatible to the solution that we require. It has all the functionality that is required to implement all of the requirements and can implement the bulk of the solution (i.e. collaborative filtering) with ease. Django is also designed to work closely with a database and also contains many other features such as tagging which makes it an ideal language to use for the solution.
We have decided to create the database in a simple MySQL style as we are all familiar with this implementation and also MySQL provides all the functionality required for the proposed solution. Django also works very closely with MySQL structured databases.
\newline
\newline
\textbf{Operating Systems}
\newline
\newline
If we are to use the proposed language (Django), operating system portability should not be an issue as Django is a multi-platform language, although the solution will initially be intended to be used on the Microsoft Windows platform. This is another advantage of using this language.
We have decided to initially aim the solution to be developed for the Microsoft Windows platform as it currently has the highest market-share and is therefore the most popular Operating System. This will allow us to have a larger potential customer base if the solution were to be released. 
\newline
\newline
\textbf{Additional Hardware/Software}
\newline
\newline
The solution is to be an online web-based application with a database backing. This being true, it will be necessary for a server to be used to host the webpage. We have been granted some server –space for our application from an external Web Design Consultant who is an associate of the Technical Officer.
There are no other software requirements for the software to run.
\newline
\newline
\textbf{Browser Options}
\newline
\newline

