The project required that members familiarise themselves with forms of technology that they were unfamiliar with, for example Django (a python web framework). Learning such software required extensive learning from web tutorials and other sources like books. Naturally, most of the technical difficulties concerned Django. For example, it took a significant amount of time to learn about how to process forms using Django. When users select an ingredient from the drop down menu, knowledge of Django forms was essential in retrieving relevant information from the database about the recipe. 

Additionally, members were required to learn new concepts like collaborative filtering and extreme programming. This was namely done by immersing in web research and group discussion. The Project Supervisor also helped in the clarification of key concepts by delivering a series of discussions during formal meetings.

The team has a clear management structure which has prevented any issues. Moreover, this enables members to take responsibility of their areas, thereby ensuring accountability and alleviating conflicts. This also ensures an equitable distribution of workload.
