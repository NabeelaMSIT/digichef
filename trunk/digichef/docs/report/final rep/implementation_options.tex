\subsection{Summary of Project Description and Specification}

The project description implies the implementation of some form of a database, either online (i.e. on a website) or offline (i.e. run on a local machine as an executable program). There are no specific details as to which language, layout or structure etc, are to be used when creating the solution. There are also no details as to which Operating System the solution should be created for and whether additional software/hardware is allowed. Considering this we have ‘taken it open ourselves’ to discuss and choose what we thought was a suitable target platform and have also discussed availability of software/hardware needed to create the solution.
During an initial meeting with our client concerning the problem specification/requirements it has become clear that the preferred solution is to create a website with a database backbone. This can be implemented in many different styles/languages which we have also discussed extensively.
 
We have created the Problem Specification (section~\ref{sec:productspec} on page~\pageref{sec:productspec})
% You can stick labels to sections and have it update the page number if it changes!
 with the intent that at each stage, the structure and format of the solution allows successive stages to be completed without changing the entire structure too much. This prototyping adheres to a large extent with the concept of Extreme Programming, where frequent releases introduce checkpoints where new customer requirements can be adopted. Additionally, the focus will be on upgrading versions of prototypes. To accommodate this dynamic character of our project, we need a framework which ties in the database with all the different elements of the website.

